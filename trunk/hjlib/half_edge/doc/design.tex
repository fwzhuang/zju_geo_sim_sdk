\documentclass[9pt,twocolumn]{extarticle}

\usepackage[hmargin=0.5in,vmargin=0.75in]{geometry}
\usepackage{parskip}
\usepackage{amsmath,amssymb}
\usepackage{times}
\usepackage{cleveref}
\Crefname{figure}{Fig.}{Figs.}
\Crefname{equation}{Eq.}{Eqs.}
\Crefname{table}{Tab.}{Tabs.}

\usepackage{color}
\newcommand{\TODO}[1]{\textcolor{red}{#1}}
\renewcommand{\ttdefault}{lmtt}
\newcommand{\CODE}[1]{\textcolor{blue}{\tt #1}}
\newcommand{\SET}[1]{\mathbf{#1}}
\DeclareMathOperator{\VAL}{val}

\title{Half-Edge based 2D Mesh Library}
\author{Jin HUANG}
\begin{document}

\maketitle

\begin{abstract}
  2D mesh is one of the fundamental entity in computer geometry.  For
  2D polygonal manifold mesh, the idea of half-edge data structure is
  famous because of the great balance on efficiency and versatility.
  This documents gives a c++ template based design for the
  implementation.
\end{abstract}

\section{Basic Concepts}
We build up the most related concepts from scratch.

We denote $n$ dimensional finite {\em coordinate chart} (not a {\em
  manifold}), as $\mathbb{C}^n$.  The element in $\mathbb{C}^0,
\mathbb{C}^1, \mathbb{C}^2$ is names as {\em point}, {\em edge} and
{\em face} respectively.  Notice here, edge can be curve and face can
be non-planar.

The boundary of a face $f$, i.e. $\partial f$, is an 1-manifold (not a
chart). Giving a set of 1-chart $\{e_i\}, i=1\cdots,n$, if
\begin{equation}
  \partial f \subset \bigcup e_k, \mbox{ and } e_i\cap e_j \in \{\emptyset,  \mathbb{C}^0\}, i \neq j,
\end{equation}
$e_i$ is the edge of $f$.  The boundary of an edge is composed of two
points.  A face with $n$ edges and the associated $n$ points is named
as {\em $n$-side polygon}.

A general mesh $M$ is a union of points, edges and polygons. If $M$ is a 2-manifold,
which possibly with boundary, we call it manifold mesh.  If all the
polygons are $3$-side, it a triangular mesh.  The most widely used
(2D) mesh is manifold triangular mesh.

Position (or coordinate) of a point, normal of a face, etc. are
properties optionally attached to the mesh.

An operation on a mesh includes two parts: the operation on the
topological structure (point, edge, face) and the consistent operation
on the associated properties.

\section{Feature of the Library}

This library will focus on utilizing half-edge data structure to
provide flexible and robust mesh operations.  Flexibility means using
half-edge to represent as-general-as-possible mesh, including
non-manifold mesh.  Robust means using rigid mathematical analysis
to guide the mesh operations.

\section{General Mesh Analysis}

A mesh $\mathcal{M}$ can be represented by $(V,E,F)$, which $V$ is the
set of vertices, $E$ is the set of directed edges, and $F$ is the set
of directed faces. A mesh can be manifold or non-manifold.  The
discussion is shown below.  For simplicity, edge stands for directed
edge in the following text.

\subsection{Basic Definitions}

%% For non-manifold mesh, an edge can belong to many faces, and one vertex can 
%% belong to many boundary edges.

%% The vertices set have some different vertices.
%% \begin{equation}
%%   V=\{v_i \ |\ i\in \mathbb{Z},0\leq i<nv\}
%% \end{equation}
%% where, $nv$ is the number of vertices.

%% The edges set have some directed edges:
%% \begin{equation}
%% E=\{ e_i \ |\ i\in \mathbb{Z}, 0\leq i < ne \}
%% \end{equation}
%% where, $ne$ is the number of edges. 

Giving the three set $\SET{V},\SET{E},\SET{F}$, half-edge data
structure could be viewed as a mappings/functions set $\SET{H}$
including the following elements:
\begin{equation}
  \begin{split}
    I &:\SET{X}\rightarrow \SET{X}, \quad \SET{X} \in \{\SET{V}, \SET{E}, \SET{F}\}\\
    e_v &: \SET{V}\rightarrow \SET{E}\cup\{\emptyset\}\\
    v_e &: \SET{E}\rightarrow \SET{V}\\
    f_e &: \SET{E}\rightarrow \SET{F}\cup\{\emptyset\}\\
    n_e, p_e &: \SET{E}\rightarrow \SET{E}\cup\{\emptyset\}\\
    o_e &: \SET{E}\rightarrow \SET{E}\\
    e_f &: \SET{F}\rightarrow \SET{E},\\
  \end{split}
\end{equation}
where $I$ is the identity mapping.  Notice here that $v_e, e_f$ cannot
map to $\emptyset$.

\subsection{$v_e$ and valence}
The result of applying $v_e$ to an edge $e$ is the vertex that points
to.

{\em Valence} of a vertex $v$ is defined as:
\begin{equation}
  \VAL(v)=|\{v_e(e)=v\}|
\end{equation}

vertex can be classified into several types according to its valence:
\begin{equation}
  \VAL(v)
  \begin{cases}
    =0&\text{isolated},\\
    =1&\text{end},\\
    =2&\text{1d-manifold},\\
    >2&\text{1d-non-manifold}.
  \end{cases}
\end{equation}

\subsection{$o_e$ and undirected edge}
We ask $o_e$ has the same property with the following additional
requirement:
\begin{equation}
  o_e = o_e^{-1}.
  \label{eq:o_e}
\end{equation}
Then, an {\em undirected edge} can be defined a set $e = \{e_+, e_-\},
e_+,e_- \in \SET{E}, o_e(e_+)=e_-$.  Its end vertex set is
$\{v_e(e_+),v_e(e_-)\}$.  If $v_e(e_+) = v_e(e_-)$, it is a loop in
the undirected graph formed by the undirected edges.

Two different undirected edges $a,b$ may have the same end
vertices, which could be used to represent two-side hole and face.

Indeed, there are two strategies for $o_e$ on boundary: no opposite,
or opposite has not face.  The second strategy has the advantage of
querying the one-ring of a non-manifold vertex.  Thus, we adopt the
second one, and name the half edge with no face as boundary edge.

\subsection{$e_v$}
It satisfies the following conditions:
\begin{equation}
  \begin{split}
    \VAL(v) = 0 &\Leftrightarrow e_v(v) = \emptyset,\\
    \VAL(v) > 0 &\Leftrightarrow
    e_v(v) \neq \emptyset \Rightarrow v_e(e_v(v)) = v.
  \end{split}
  \label{eq:e_v}
\end{equation}
Notice here that $e_v(v_e(e)) = e$ is not generally hold.

NOTE: To specify a boundary vertex efficiently, we guarantee that $e_v$ of a boundary vertex is a boundary edge. So when modify topology using low-level functions, be sure to make this feature hold.

\subsection{$n_e,p_e$}
\TODO{A strong requirement}:
\begin{equation}
  \begin{split}
    n_e(e) = \emptyset \Leftrightarrow \VAL(v_e(e)) = 1,\\
    n_e(e) \neq \emptyset \Leftrightarrow \VAL(v_e(e)) > 1.
  \end{split}
  \label{eq:val_n}
\end{equation}

\TODO{For intuition}, for all $e \in \SET{E}$ we ask
\begin{equation}
  \begin{split}
    n_e(e) \neq \emptyset \Rightarrow e &= p_e(n_e(e)),\\
    p_e(e) \neq \emptyset \Rightarrow e &= n_e(p_e(e)).
  \end{split}
  \label{eq:np_e}
\end{equation}
Notice $n_e,p_e$ may map an edge to $\emptyset$, and thus may have no
inverse, thus $p_e = n_e^{-1}$ is not generally hold.

If $n_e(e) \neq \emptyset$ (or similarly, $p_e(e) \neq \emptyset$):
\begin{equation}
  \begin{split}
    v_e(e) &= v_e(o_e(n_e(e)))\\
    v_e(p_e(e)) &= v_e(o_e(e))\\
  \end{split}
\end{equation}

A strong requirement should be hold, i.e. using $n_e, o_e$ can visit all the adjacent edges of a vertex $v$:
\begin{equation}
  \{(p_eo_e)^ke_v(v)\} = \{e|v_e(e) = v\}.
\end{equation}
According to this requirement, if a non-manifold vertex is adjacent to
four boundary edges (see below), \TODO{it can be proved} that these
edges have uniquely defined $n_e,p_e$.


%% Two different edges $a, b$ are {\em touched} at vertex $v$ if and only
%% if:
%% \begin{equation}
%%   v_e(a) = v_e(o_e(b)) = v.
%% \end{equation}
For 1d-manifold vertex $v$, the following condition can be derived
from \Cref{eq:val_n,eq:np_e}:
\begin{equation}
  o_e(n_e(o_e(n_e(e_v(v))))) = e_v(v).
  \label{eq:1d-manifold}
\end{equation}

%% Two different edges $a, b$ are {\em connected} if
%% \begin{equation}
%%   n_e(a)=b.
%% \end{equation}
%% Notice that $v_e(a) = v_e(o_e(b)) = v$ does not mean $a,b$ are
%% connected, although $a,b$ touch each other at $v$.

%% For connected $a,b$, the following condition must hold:
%% \begin{equation}
%%   c = n_e(o_e(b)) \neq \emptyset \wedge v_e(o_e(c)) = v,
%%   \label{eq:connected}
%% \end{equation}
%% but $c = o_e(a)$ is not generally true.

\subsection{$f_e$ and $e_f$}
If and only if the edge $e$ is a boundary of a face $f$, $f_e(e)=f$,
otherwise $f_e(e)=\emptyset$.  $e_f$ returns one of the boundary edge
of face.

The following equalities must hold:
\begin{equation}
  \begin{split}
    f_e(n_e^k(e)) &= f_e(e), \quad k=0, 1, \cdots.
    \label{eq:fe}
  \end{split}
\end{equation}
Notice $e_f(f_e(e)) = e$ is not generally true.

We also ask:
\begin{equation}
  f_e(e) \neq \emptyset \Rightarrow f_e(e) \neq f_e(o_e(e)) \wedge
  \exists 0 < k \leq |\SET{E}|, n_e^k(e) = e.
\end{equation}
When $k=1$, it is a face bounded by a single curved edge.

Boundary edges are defined as:
\begin{equation}
  \partial_e \SET{F} = \left\{e|f_e(e)=\emptyset \wedge f_e(o_e(e)) \neq
  \emptyset\right\}.
\end{equation}
Notice that isolated (to faces) edges are not boundary edges.
Boundary vertex are defined as:
\begin{equation}
  \partial_v \SET{F} = \left\{v_e(e)|e\in \partial_e \SET{F}\right\}.
\end{equation}
\TODO{It can be proved that}
\begin{equation}
  v \in \partial_v \SET{F} \Rightarrow \VAL(v) \geq 2.
\end{equation}

\subsection{Equalities Constraints}
The above analysis provides a set of constraints to a valid (under
little assumptions) half edge data structure.  When a variable is fed
into a function, it must be not $\emptyset$:

\begin{equation}
  \begin{cases}
    o_e(o_e(e)) = e &\text{\Cref{eq:o_e}}\\
    \VAL(v) = 0 \Rightarrow e_v(v) = \emptyset &\text{\Cref{eq:e_v}}\\
    \VAL(v) > 0 \Rightarrow
    e_v(v) \neq \emptyset \Rightarrow v_e(e_v(v)) = v &\text{\Cref{eq:e_v}}\\
    n_e(e) \neq \emptyset \Rightarrow e = p_e(n_e(e)) &\text{\Cref{eq:np_e}}\\
    p_e(e) \neq \emptyset \Rightarrow e = n_e(p_e(e)) &\text{\Cref{eq:np_e}}\\
    \VAL(v)=2 \Rightarrow o_e(n_e(o_e(n_e(e_v(v))))) = e_v(v) &\text{\Cref{eq:1d-manifold}}\\
    f_e(n_e^k(e_f(f))) = f, \quad k=0, 1, \cdots &\text{\Cref{eq:fe}}
  \end{cases}.
\end{equation}

\section{Prev Doc}
%And every edge has opposize edge, i.e. the start and end vertex are opposite.

Define two functions from $E$ to $V$:
\begin{align}
&from(e) \rightarrow v_a\\
&to(e) \rightarrow v_b
\end{align}
where, $v_a$ is the start vertex of $e$, and $v_b$ is the end vertex of $e$.

In order to facilitate discussion, define an edge $e\in E$ as a triple:
\begin{equation}
e=\langle v_a, v_b, \kappa \rangle
\end{equation}
where, $0\leq \kappa <|CE(v_a,v_b)|$, $CE$ is defined below, the edges set with 
start vertex $v_a$ and end vertex $v_b$.
\begin{equation}
CE(v_a,v_b)=\{e \ |\ e\in E, from(e)=v_a,to(e)=v_b\}
\end{equation}

More specially, if $v_a=v_b$, $e$ is a self-loop.

If a vertex $v\in V$ don't have adjacent edges, i.e. there is no edge from $v$, $v$ is isolated.

The faces set have some directed faces:
\begin{equation}
F=\{ f_i \ |\ i\in \mathbb{Z}, 0\leq i < nf \}
\end{equation}
where, $nf$ is the number of faces. Some faces may have
common edges, i.e. $\partial f_i \cap \partial f_j \neq \emptyset$. 
And even some faces have the same edges set, i.e. $\partial f_i=\partial f_j$.
Here, $\partial f$ is the boundary edges of $f$, and $\partial f\in E$,
which is defined clearly as multi tuples below.
\begin{equation}
\begin{split}
\partial f = \{ e_{f,i}\ |\ i\in \mathbb{Z}, 0\leq i<en \}
\label{eq:boundface}
\end{split}
\end{equation}
where, $en$ is the number of edges in $f$,and
\begin{align}
&\forall i\in \mathbb{Z}, 0\leq i <en,to(e_{f,i})=from(e_{f,(i+1)mod\ en})\\
&\forall \ 0\leq a,b<en,\ e_{f,a}= e_{f,b}\ \mbox{iff}\ a=b.
\end{align}

According to above definition, give a function from $F\times \mathbb{Z}$ to $E\cup \emptyset$:
\begin{equation}
fedge(f,p)=\left\{
\begin{aligned}
&e_{f,p} &0\leq p<|f|\\
&\emptyset & \mbox{otherwise}
\end{aligned}\right.
\label{eq:fedge}
\end{equation}
where, $f\in F, p\in \mathbb{Z}, e_{f,p}\in \partial f$ defined as Eq~\ref{eq:boundface}.
$|f|$ is the valence of $f$, i.e. the edge number of $f$. 
$\partial f_i$ is the boundary edges set of $f_i$, $\partial f_i\subset E$.
$|f|=1$, the face is a self-loop. $|f|=2$, the face is two side face.

Here, define a bijective function from $E$ to $E$:
\begin{equation}
oppo(e_i) \rightarrow e_j
\end{equation}
where, $e_i=\langle v_a,v_b,\kappa\rangle, e_j=\langle v_b,v_a,\kappa\rangle, v_a,v_b\in V$, 
here $\kappa$ is defined above.

For descripting an edge adjacent many faces, here introducing a set $B$.
Now, define a set of boundary edges of faces:
\begin{equation}
B = \{ b_i\ |\ i\in \mathbb{Z}, 0\leq i<nb\}
\end{equation}
where, $nb$ is $\sum_{f\in F}|\partial f|$, $\forall b_i\in B, b_i=\langle e,f \rangle$, 
$e\in E, e\in \partial f, f\in F$. Every element of $B$ is a two tuple.

To sum it up, $E$ is based on $V$, $F$ is based on $E$, and $B$ is based on $E$ and $F$.

\subsection{More about $E$}
$\mathcal{M}$ has boundary $\partial \mathcal{M}$. And $\partial \mathcal{M}\subset E$.
For edges set $E$, it can be represented by:
\begin{equation}
E = (\bigcup_{i=0}^{|F|-1}\partial f_i) \cup \partial \mathcal{M}\cup E_I
\end{equation}
where, $E_I$ is the isolated edges set, which don't have any adjacent faces. 
And define $E_I\cap \partial \mathcal{M}=\emptyset$, i.e. boundary edges 
don't include isolated edges.

As general, the boundary edges of $\mathcal{M}$ should have the property:
\begin{equation}
\partial \mathcal{M} \cap \bigcup_{f\in F} \partial f = \emptyset
\end{equation}

Also, a property can be got:
\begin{equation}
\forall e\in E \Rightarrow \exists f\in F, e\in \partial f \mbox{ or } e\in \partial \mathcal{M}
\mbox{ or }e\in E_I.
\end{equation}

\subsection{Some Useful Functions}

Define a function from $E$ to $V$:
\begin{equation}
verts(e)\rightarrow \{v_i,v_j\}, \mbox{ if }e=\langle v_i,v_j,\kappa \rangle \in E
\end{equation}
where, $v_i,v_j\in V$. Here, define $\partial e = verts(e)$.

Define a function from $F$ to $V$:
\begin{equation}
verts(f)\rightarrow \{v_k\ |\ k\in \mathbb{Z},
0\leq k<|f|, v_k=e_{f,k}\}
\end{equation}
where, $f\in F$, $e_{f,k}$ is defined in Eq~\ref{eq:fedge}.

Define a function from $E$ to $B\cup \emptyset$:
\begin{equation}
bedge(e)=\left \{
\begin{aligned}
&b& \mbox{if } \exists f\in F,\mbox{ s.t }
e\in \partial f, b=\langle e,f \rangle\\
&\emptyset & \mbox{otherwise, i.e. }e\mbox{ doesn't have adjacent faces.}
\end{aligned}\right.
\end{equation}
where, $e\in E, b\in B$. The function is surjective.

Define a function from $B$ to $E$:
\begin{equation}
edge(b) \rightarrow e
\end{equation}
where, $b\in B, e\in E$, and $b=\langle e,f \rangle$. 
The function is surjective if $E$ doesn't have isolated edges,
which don't have adjacent faces.

For set $B$, define a function from $B$ to $B$:
\begin{equation}
next(b_i) \rightarrow b_j
\end{equation}
where, $\exists f\in F, p\in \mathbb{Z}, 
\mbox{ s.t. } edge(b_i)=e_{f,p},edge(b_j)=e_{f,(p+1)mod|f|}$, $e_{f,p}\in \partial f$, 
$|f|$ is the number of edges in $f$. The function is bijective.

Here, define a similar function from $B$ to $B$:
\begin{equation}
prev(b_i) \rightarrow b_j, \mbox{ if }next(b_j)=b_i.
\end{equation}

Now, define a function from $B$ to $F$:
\begin{equation}
face(b) \rightarrow f
\end{equation}
where, $b=\langle e,f \rangle, f\in F, e\in E, b\in B$. The function 
is surjective.

For an edge $e\in E$, there are many $b\in B$, so that $edge(b)=e$. Here, define:
\begin{equation}
BE(e)=\{b\ |\ b\in B, edge(b)=e \}
\end{equation}

Define a function from $B$ to $B$:
\begin{equation}
down(b_i)=
\left\{
\begin{aligned}
&b_j & \mbox{if }|BE(edge(b_i))|>1\\
&b_i & \mbox{otherwise}
\end{aligned}
\right.
\end{equation}
where, $face(b_i)=face(b_j)=f\in F, edge(b_i)=edge(b_j)=e\in E$, 
and $\exists p\in \mathbb{Z}, 0\leq p<|BE(e)|$,
s.t. $b_i=BE(e)_p,b_j=BE(e)_{(p+1)mod\ |BE(e)|}$, $BE(e)_i$ is the $i$th element of $BE(e)$.

Here, define a function similar to $down$ from $B$ to $B$:
\begin{equation}
up(b_i) \rightarrow b_j, \mbox{ if }down(b_j)=b_i.
\end{equation}

Now, there is no function from $V$ to other sets. For query and mesh operation,
this type function should be necessary. Here, define a function from $V$ to $E$:
\begin{equation}
vedge(v)=
\left\{
\begin{aligned}
&e& \mbox{ if }\exists e\in E, from(e)=v\\
&\emptyset & \mbox{otherwise, i.e. an isolated vertex}
\end{aligned}\right.
\end{equation}

For the discussion that follows, define a set:
\begin{equation}
VE(v)=\{e\ |\ e\in E, from(e)=v\}, v\in V
\end{equation}

Sometimes, from a vertex the adjacent edges should be got. To get it, define
a function from $E$ to $E$:
\begin{equation}
cnext(e_i) =\left\{
\begin{aligned}
&e_j& \mbox{if } |VE(from(e_i))|>1\\
&e_i& \mbox{otherwise}
\end{aligned}\right.
\end{equation}
where, if $v=from(e_i),|VE(v)|>1$, $\exists p\in \mathbb{Z}, 0\leq p<|VE(v)|$, 
s.t. $e_i=VE(v)_p,e_j=VE(v)_{(p+1)mod\ |VE(v)|}$, here $VE(v)_i$ is the $i$th element
of $VE(v)$.

Now define the other function from $E$ to $E$:
\begin{equation}
cprev(e_i)\rightarrow e_j, \mbox{ if } cnext(e_j)=e_i.
\end{equation}

\subsection{More Discussion About $B$}
Set $B$ can represent $F$. By $next$ or $prev$ can get all faces' edges.
$\forall b\in B, [b]_{next}$, which is a equality based on $next$, can represent a face $f\in F$.

Below discussion, use a tretrad $(V,E,F,B)$ to represent $\mathcal{M}$.

\subsection{Non-manifold Edge}

If an edge has more than one adjacent faces or don't have adjacent face, it
is a non-manifold edge:
\begin{itemize}
\item $bedge(e)=\emptyset,bedge(oppo(e))=\emptyset$. {\color{red} Maybe need to discuss}
\item $bedge(e)\neq\emptyset,up(bedge(e))\neq e$.
\item $bedge(oppo(e))\neq \emptyset, up(bedge(oppo(e)))\neq oppo(e)$.
\end{itemize}

If an edge is a manifold edge:
\begin{itemize}
\item $bedge(e)\neq\emptyset,up(bedge(e))= e,bedge((oppo(e)))=\emptyset$.
\item $bedge(e)\neq\emptyset$, $up(bedge(e))= e$,
$bedge((oppo(e)))\neq \emptyset$, $up(beddge((oppo(e))))= oppo(e)$.
\item $bedge(e)=\emptyset$, $bedge(oppo(e))\neq \emptyset$, $up(bedge(oppo(e)))= oppo(e)$.
\end{itemize}

\subsection{Non-manifold Vertex}

\begin{itemize}
\item A vertex belong to an non-manifold edge.
\item A vertex is a pinched vertex.
\item A vertex is an isolated vertex, $edge(v)=\emptyset$. 
{\color{red} Maybe need to discuss}
\end{itemize}

\subsection{How to check manifold}

According to above two sections, the manifold mesh can be checked easily.

\subsection{Boundary Of Mesh}
\paragraph{Boundary Edges} $e\in E$ is a boundary edge $e\in \partial \mathcal{M}$,
if $bedge(e)=\emptyset$. A boundary edge can have more than one adjacent faces.
\paragraph{Boundary Vertex} $v\in V$ is a boundary vertex, if
\begin{equation}
\exists e\in E, from(e)=v, bedge(e)=\emptyset.
\end{equation}
\paragraph{Boundary Face} $\forall \ f\in F$, if 
$\exists e \in \partial f\ \mbox{and}\ oppo(e) \in \partial \mathcal{M}$, 
$f$ is a boundary face.


\subsection{Hole Of Mesh}
For non-manifold mesh, holes can not be found. Then, holes discussion followed 
are ones that can be got exactly without ambiguity.

Here, define a set representing holes in $\mathcal{M}$:
\begin{equation}
H=\{h_i\ |\ i\in \mathbb{Z}, 0\leq i<nh\}
\end{equation}
where, $nh$ is the number of holes in $\mathcal{M}$.

For a hole $h\in H$,
\begin{equation}
\partial h\in \partial \mathcal{M}, \partial h\in E
\end{equation}
where, $\partial h$ is the edges set of $h$.

\paragraph{How to find a hole in $\mathcal{M}$} A hole can be found from
an boundary edge.

\subsection{Add}
\paragraph{Add Vertex} When adding a vertex to $\mathcal{M}$, a new mesh can be got:
\begin{equation}
\mathcal{M}'(V\cup\{v_a\},E,F,B)=\mathcal{M}(V,E,F,B)+v_a
\end{equation}

\paragraph{Add Edge} When adding an edge to $\mathcal{M}$, a new mesh can be got:
\begin{equation}
\mathcal{M}'(V',E',F,B)=\mathcal{M}(V,E,F,B)+e_a
\end{equation}
where,
\begin{align}
&V'=V\cup verts(e)\\
&E'=E\cup \{e_a,oppo(e_a)\}
\end{align}

\paragraph{Add Face} When adding a face to $\mathcal{M}$, a new mesh can be got:
\begin{equation}
\mathcal{M}'(V',E',F',B')=\mathcal{M}(V,E,F,B)+f_a
\end{equation}
where,
\begin{align}
&V'=V\cup verts(f_a)\\
&E'=E\cup \partial f_a \cup oppo(\partial f_a)\\
&F'=F\cup \{f_a\}\\
&B'=B\cup \{\langle e,f_a \rangle\ |\ e\in \partial f_a\}
\end{align}
where, $oppo(\partial f_a)=\bigcup_{e\in \partial f_a}oppo(e)$.

To sum it up, add operation can be subdivided into several steps.
For add face operation, it can be broken into three steps: add vertices, add edges
and add face. For add edge operation, it can be broken into two steps: add vertices 
and add edges.

\subsection{Delete}

\paragraph{Delete Face} When deleting a face in $\mathcal{M}$, a new mesh can be got:
\begin{equation}
\mathcal{M}'(V,E,F',B')=\mathcal{M}(V,E,F)-f_d
\end{equation}
where,
\begin{align}
&F'=F-\{f_d\}\\
&B'=B-\{\langle e,f_d \rangle \ |\ e\in \partial f_d \}
\end{align}

\paragraph{Delete Edge} When deleting an edge in $\mathcal{M}$, a new mesh can be got:
\begin{equation}
\mathcal{M}'(V',E',F',B')=\mathcal{M}(V,E,F,B)-e_d
\end{equation}
when deleting an edge, the opposite edge also should be deleted in order to keep 
$oppo$ function bijective, so,
\begin{align}
&V'=V\\
&E'=E-\{e_d,oppo(e_d)\}\\
&F'=F-face(BE(e_d))\cup face(BE(oppo(e_d)))\\
&B'=B-BE(e_d)-BE(oppo(e_d))
\end{align}
where,
\begin{equation}
face(BE(e_d))=\bigcup_{b\in BE(e_d)}face(b).
\end{equation}

\paragraph{Delete Vertex} 
When deleting a vertex in $\mathcal{M}$, a new mesh can be got:
\begin{equation}
\mathcal{M}'(V',E',F')=\mathcal{M}(V,E,F)-v_d
\end{equation}
where,
\begin{align}
&V'=V-\{v_d\}\\
&E'=E-VE_{v_d}\cup oppo(VE_{v_d})\\
&F'=F-face(BE(VE_{v_d}))\cup face(BE(oppo(VE_{v_d})))\\
&B'=B-BE(VE_{v_d})\cup BE(oppo(VE_{v_d}))
\end{align}
here,
\begin{equation}
oppo(VE_{v_d})=\bigcup_{e\in VE_{v_d}}oppo(e).
\end{equation}

To sum it up, every delete operation can be subdivided into several parts.
For deleting vertex, it can be broken into three parts: delete faces,
delete edges, and delete vertex. For deleting edge, it can be broken
into two parts: delete faces, delete edges.

\subsection{Query}

Query edges adjacent of vertex: using $cnext$ and $cprev$.

Query faces adjacent of vertex: using $cnext$, $cprev$,$bedge$ and $face$.

Query faces adjacent of edges: using $bedge$ and $face$.

\subsection{Merge vertices}

Here, discuss an operation $merge(\mathcal{M}(V,E,F,B),e), e\in E$, 
which merge $from(e)$ to $to(e)$.

The new mesh will be:
\begin{equation}
\mathcal{M}'(V',E',F',B')=merge(\mathcal{M}(V,E,F,B),e).
\end{equation}
where,
\begin{align}
&V' = V-\{from(e)\} \\
&E' = E-\{e\}\\
&F' = F-face(BE(e))\cup face(BE(oppo(e)))\\
&B' = B-BE(e)\cup BE(oppo(e))
\end{align}
After merge, $next$ and $prev$ from $B$ to $B$ should be updated.

\section{Manifold Analysis}

A manifold\footnote{Around every point of mesh, there is a neighborhood homeomorphic 
disk or half-disk.} mesh is useful in many applications of computer graphics 
because it has many good properties.

\subsection{Some Properties}
A manifold mesh has some properties:
\begin{itemize}
\item Every edge has one or two adjacent faces expect isolated edges. 
If with two, these two faces have the same 
direction.
\item Every vertex has at zero or two boundary edges expect isolated vertices.
\end{itemize}

The first condition can be written as:
\begin{equation}
\partial f_i \cap \partial f_j = \emptyset, \forall \ f_i,f_j\in F,\ i\neq j
\label{eq:edgecondition}
\end{equation}
Above condition cannot deduce that $\mathcal{M}$ is a manifold.

The second condition can be written as:
\begin{equation}
\begin{split}
&BV(v)=\{e\ |\ e\in E, from(e)=v,bedge(e)=\emptyset\},\\
&|BV(v)|<2.
\end{split}
\end{equation}

If there are many holes in mesh $\mathcal{M}$, define them as:
\begin{equation}
H=\{h_i\ |\ i\in \mathbb{Z}, 0\leq i<nh \}
\end{equation}
where, $nh$ is the holes number in $\mathcal{M}$. $h_i$ is one hole of $\mathcal{M}$.

According above, The below can be got:
\begin{align}
&\partial \mathcal{M} = \bigcup_{h\in H} \partial h\\
&\partial h_i\cap \partial h_j = \emptyset, \forall h_i,h_j \in H, h_i\neq h_j\\
&\partial h \cap \partial f = \emptyset, \forall h\in H, f\in F
\end{align}
where, $\partial h_i$ is the edges set of hole $h_i$.

\subsection{Flip Edge}
This operation is only used in triangle mesh.

% {\color{red}
% When fliping an edge $e$, $e$ and $oppo(e)$ should not be a boundary edge, 
% i.e. $e,oppo(e) \notin \partial \mathcal{M}$.

% Define $f_a=face(e),f_b=face(oppo(e)), f_a\neq \emptyset, f_b\neq \emptyset$.
% Here, $|f_a|=|f_b|=3$, i.e. they are triangles.

% Define $e_o= oppo(e), e_n=next(e), e_{op}=prev(oppo(e)), e_p=prev(e), e_{on}=next(oppo(e))$.

% Define $v_a=from(e), v_b=to(e), v_c=to(e_n), v_d=from(e_{op})$.

% When $\exists e\in E, v_c=from(e),v_d=to(e)$, $e$ can not be fliped, which can lead to
% edges with the same vertices.

% For fliping an edge, the two faces should have new corresponding edges:
% $f_a=\langle e_o,e_{op},e_n \rangle,f_b=\langle e,e_p,e_{on} \rangle$.

% Now, according to the new faces, the new $next$ and $prev$ about the two faces can be got.}

\subsection{Collapse Edge}



\subsection{Split Edge}


\section{Implementation}
We introduce the basic idea of programming techniques used in this
implementation, then introduce how to apply them into the design.

{\em Policies} are classes (or class templates) to inject behavior
into a class.  {\em Traits} are class templates to extract properties
from a generic type.

For efficiency, the library will provide traits and policies for
triangular mesh, quadrilateral mesh and hybrid mesh.  The traits
should be able to extended by users for specific requirements.

The properties will be highly separated from the topological
structure, and can have flexible data structure for efficiency and
simplicity.

A set of operations, which is isolated to traits and property data
structure, should be provide from very basic ones, which may not
preserve manifold property, to high level ones, which provides some
typical algorithms.

\subsection{Implementation Analysis and Design}

Half-edge is mainly used for a mesh with changing topology,
i.e. adaptivity, simplification etc.  The efficiency of this data
structure mainly comes from locality, i.e. minimal query and change
for an atom operation.

The same type of entities in a mesh should be put into a container,
and link to each other.  There are several possible way to implement
the link.
\begin{itemize}
\item Pointer or iterator: directly refer to the object, but will make
  it difficult to bind with external property.
\item Hashing: through an auxiliary mapping, i.e. index to address
  etc.  Additional cost, but can support external property easier.
\end{itemize}

OpenMesh uses handle (index into an array), which needs an global
query (array[index]) to get the entity. Such a strategy has two
problems: first, the container (array) will be provided for every
query, thus, one more argument is required (in OpenMesh, it is
\CODE{this}); it needs a global query, which may be costly for other
containers, like map.  We prefer the strategy of CGAL, which uses
iterator although they give it an alias of handle.  To use iterator,
to detect an invalid iterator, CGAL use customized iterator (ref:
\verb|HalfedgeDS_iterator_adaptor.h|).  Indeed, the iterator
constructed using default constructor looks good as a ``null''.

Another issue about handle and iterator is in function call with const
decorator.  Some function just collects information of a mesh, thus
the input mesh should be const.  Using iterator, it can only return
const-iterator.  One possible method is get iterator from
const-iteartor, but it breaks the const way.

A solution is use the address as the handle, then we can have both
advantage with the defect of using cast. Or use const-iterator as
handle, and every modification pass the const-iterator into the
instance, i.e. instance can cast const-iterator into iterator.

An entity is associated with a handle, which has no ``++'' etc,
operator, but should be orderable.

Query is the most important operation for mesh.  It can be classified
as {\em local} query and {\em global} query.  Local query means to
access the neighboring (according to the half-edge) entities from the
current one, and the global query means giving a handle in the mesh to
fetch the corresponding one.  Global query is necessary, even only for
topological part, e.g. delete an entity, which must inform the pool
(container).  Thus handle is difficult to avoid.  Usually the handle
is the iterator of container.

  Usually, property query is global.
Local query should has complexity $O(1)$. However, global one may be
more costly, and needs handle, thus should be used as less as
possible.  Thus, the properties can be classified as ``internal''
properties which are inside of the entity and ``external'' ones which
are outside of the entity.

Based on the above analysis, using memory address for the neighboring
entities is better than using handle (like OpenMesh and CGAL).  The
issue is that there is not handle intrinsically defined in this data
structure, and global query looks difficult.  There are two possible
solutions: if the internal storage is array like, handle can be
defined from address, otherwise, we can use property.


One of the most important question is how to connect entity to
property.  One of the trivial way is to build property into entity,
but this strategy lacks flexibility and usually leads to high cost.
Another way is to use handles (e.g. index) associated to the entities
to query the properties.  A common way (like OpenMesh) intrinsically
include handle according to entities

Conceptually, the objects in the library include:
\begin{itemize}
\item entity: point, half-edge and face.
\item handle: non-serializable unique key to the entity in a same
  type in a mesh.
\item entity pool:  provides mapping from handle to entities.
\item property pool: provides mapping from handle to properties.
\item mesh: composed of the entity pool, property pool, and related
  policy and traits.
\item operators: applying non-constant operations on mesh.
\end{itemize}

Handle, may not be the index of entities.  They may not continuously
fill the rage from $0$ to $n$.  We can use property to maintain the
required index, or, alias of entity.  A simplest design of handle is
the memory address of the entity, but leaves the detection of invalid
difficult.


{\em Pool} (for both entity and property) can be in different types as
shown in \Cref{tab:pool}, which is the key of the library. \TODO{To
  verify: OpenMesh uses array with tag (removal support), CGAL uses
  array (no removal support) or list (removal support).}
\begin{table}[htb]
  \begin{center}
    \begin{tabular}{|c|cc|}
      \hline
      type&query&modify\\\hline
      array&$1$&$N$\\\hline
      hash$^*$&1&1\\\hline
      list&$N$&$1$\\\hline
      map&$\log(N)$&$\log(N)$\\\hline
    \end{tabular}
  \end{center}
  \caption{Types of pool. $^*$ is customized to use more space for
    efficiency.}
  \label{tab:pool}
\end{table}

The operations will be stacked by layers. The 0-layer provides atom
operations which may even break the manifold property, i.e. insert a
dangling point.  The 1-layer provides atom operations which keep
manifold property, i.e. edge flip. The 2-layer implements some typical
algorithms.

\section{Evaluation}

\section{Conclusion}
\end{document}

